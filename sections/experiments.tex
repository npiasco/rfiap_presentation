\section{Experiments}
\begin{frame}{Robotcar dataset}
	\begin{figure}[t]
		\centering	
		\only<1>{\includegraphics[width=0.8\linewidth]{images/map/map.png}  }
		\only<2>{
			\begin{minipage}{0.2\linewidth}
				\centering
				Query
				\vspace{1cm}
				
				Closest example in the database			
			\end{minipage}
			\hfill
			\begin{minipage}{0.79\linewidth}
				\includegraphics[width=0.24\linewidth]{images/ex_query/query0.jpg}\hfill
				\includegraphics[width=0.24\linewidth]{images/ex_query/query2.jpg}\hfill
				\includegraphics[width=0.24\linewidth]{images/ex_query/query3.jpg}\hfill
				\includegraphics[width=0.24\linewidth]{images/ex_query/query4.jpg}
			
				\includegraphics[width=0.24\linewidth]{images/ex_query/dataset0.jpg}\hfill
				\includegraphics[width=0.24\linewidth]{images/ex_query/dataset2.jpg}\hfill
				\includegraphics[width=0.24\linewidth]{images/ex_query/dataset3.jpg}\hfill
				\includegraphics[width=0.24\linewidth]{images/ex_query/dataset4.jpg}
			\end{minipage}
		}
	\end{figure}
	
	\only<1>{Dataset training (green), validation (blue) and testing (red) areas.}
	\only<2>{Examples of queries with corresponding dataset candidates of the testing set.}
\end{frame}

\begin{frame}{Building dense modality map}
	\begin{figure}[t]
			\newcolumntype{Y}{>{\centering\arraybackslash}X}
			\centering
			\begin{footnotesize}
			\begin{tabularx}{\linewidth}{Y Y Y}
				\textbf{Image}	  & \textbf{Points cloud} 			& \uncover<2>{\textbf{Dense depth map}} \\
			\end{tabularx}
			\end{footnotesize}
			\includegraphics[width=0.33\linewidth]{images/dense_map_creation/image0.jpg}\hfill
			\includegraphics[width=0.33\linewidth]{images/dense_map_creation/sparsedepth0.png}\hfill
			\uncover<2>{\includegraphics[width=0.33\linewidth]{images/dense_map_creation/densedepth0.jpg}}\hfill
			
			\includegraphics[width=0.33\linewidth]{images/dense_map_creation/image1.jpg}\hfill
			\includegraphics[width=0.33\linewidth]{images/dense_map_creation/sparsedepth1.png}\hfill
			\uncover<2>{\includegraphics[width=0.33\linewidth]{images/dense_map_creation/densedepth1.jpg}}\hfill
	\end{figure}
	
	We use the algorithm proposed in~\cite{Bevilacqua2017} to create a dense modality map from an image and the associated point cloud.
\end{frame}

\begin{frame}{Results}
	\begin{figure}[t]
		\centering % TODO: update the axes
		\includegraphics[width=0.499\linewidth]{images/global_res/recall2.jpg}\hfill
		\includegraphics[width=0.499\linewidth]{images/global_res/dist2.jpg}			
	\end{figure}	
	Off-the-shelf: network only trained on ImageNet, no fine-tuning for this specific task and on these specific data.
\end{frame}

\begin{frame}{Results - Diversification loss}
	\begin{figure}[t]
		\centering % TODO: upadte the axes
		\includegraphics[width=0.499\linewidth]{images/diffloss_res/recall2.jpg}\hfill
		\includegraphics[width=0.499\linewidth]{images/diffloss_res/dist2.jpg}
	\end{figure}	
\end{frame}

\begin{frame}{Results - Visual inspection}
	\begin{minipage}[c][0.65\textheight]{0.19\linewidth}
		Main modality 
		\vspace{1.5cm}		
		
		Ground truth side modality
		\vspace{1.5cm}
		
		Reconstructed side modality
	\end{minipage}\hfill
	\begin{minipage}[c][0.7\textheight]{0.75\linewidth}
		\includegraphics[width=0.95\linewidth]{images/visual_results/mod_rgb.png}
		\vfill		
		\includegraphics[width=0.95\linewidth]{images/visual_results/gt_depth.png}
		\vfill		
		\includegraphics[width=0.95\linewidth]{images/visual_results/reconstructed_maps.png}
	\end{minipage}	
\end{frame}

\begin{frame}{Results - Conclusion}
	Comparison with state of the art method:
	\vfill
	\begin{tabular}{c | c | c | c}
						&	Improvement over 	& 	\multirow{2}{*}{Size} 	& 	\multirow{2}{*}{Training steps}			 \\
						&	images-only method  	&  	&  \\
		\hline				
		\hline						
		Hallucination	& 	\uncover<2->{\multirow{2}{*}{\textbf{Yes}}}			& \uncover<3->{\multirow{2}{*}{2x}}	& \uncover<4>{\multirow{2}{*}{3 steps}}	 \\
		\cite{Hoffman2016} & & & \\
		\hline						
		BU-TF	& \uncover<2->{\multirow{2}{*}{\textbf{Yes}}}			& \uncover<3->{\multirow{2}{*}{\textbf{1.4x}}}	& \uncover<4>{\multirow{2}{*}{\textbf{2 steps}}} \\
		(Our proposal) & & & \\
	\end{tabular}
	\vfill
\end{frame}